\documentclass[12pt]{article}

\usepackage[a4paper,margin=2.5cm,footskip=0.7cm,headheight=1cm]{geometry}

\usepackage{amsmath}
\usepackage{mathtools}
\usepackage{scalerel,amssymb}

\newcommand{\overtext}[2]{\mathrel{\overset{\makebox[0pt]{\mbox{\normalfont\tiny\sffamily #2}}}{#1}}}
\newcommand{\comment}[1]{}
\DeclarePairedDelimiter\abs{\lvert}{\rvert}
\DeclarePairedDelimiter\brackets{(}{)}

%TODO
\title{\vspace{-2.0cm}Aufgabe 1}
\author{Nikolas Kilian}
\date{8. März 2019}

\usepackage{lastpage}
\usepackage{fancyhdr}
\pagestyle{fancy} 

\makeatletter
\let\runauthor\@author
\makeatother
\lhead{\runauthor}
\cfoot{\thepage\ of \pageref{LastPage}}

\begin{document}
\maketitle

\section{Lösungsidee}
\subsection{Reduktion der Möglichkeiten}
\begin{itemize}
\item Alle nicht geraden Wege sind suboptimal\\
\item Alle Wege die nicht direkt an Polygonecken oder den Buspfad gehen sind suboptimal\\
\item Der ununterbrochene Weg zum Buspfad muss im 30$\deg$ Winkel sein\\
\end{itemize}
Um die Nummer der Möglichkeiten zu reduzieren sucht man von allen Eckpunkten alle anderen erreichbaren Eckpunkte. 

\end{document}