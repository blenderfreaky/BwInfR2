\documentclass[12pt]{article}

\usepackage[a4paper,margin=2.5cm,footskip=0.7cm,headheight=1cm]{geometry}

\usepackage{amsmath}
\usepackage{mathtools}
\usepackage{scalerel,amssymb}
\usepackage{gensymb}

\newcommand{\overtext}[2]{\mathrel{\overset{\makebox[0pt]{\mbox{\normalfont\tiny\sffamily #2}}}{#1}}}
\newcommand{\comment}[1]{}
\DeclarePairedDelimiter\abs{\lvert}{\rvert}
\DeclarePairedDelimiter\brackets{(}{)}

%TODO
\title{\vspace{-2.0cm}Aufgabe 1}
\author{Nikolas Kilian}
\date{8. März 2019}

\usepackage{lastpage}
\usepackage{fancyhdr}
\pagestyle{fancy} 

\makeatletter
\let\runauthor\@author
\makeatother
\lhead{\runauthor}
\cfoot{\thepage\ of \pageref{LastPage}}

\begin{document}
\maketitle

\section{Lösungsidee}
Wenn es keine Hindernisse gibt, so ist der optimale Weg eine gerade Strecke vom Startpunkt zum Buspfad im 30\degree\ Winkel.
Gibt es Hindernisse, so ist der optimale Weg der optimale Weg zu einem Eckpunkt, von dem die 30\degree\ Strecke offen ist, und dann diese 0\degree\ Strecke.
Um das Optimum mit Hindernissen zu finden, muss man also alle Eckpunkte bestimmen, von denen aus diese 30\degree\ Strecke offen ist, und den optimalen Weg zu ihnen bestimmen. Unter den resultierenden Wegen ist das Optimum enthalten, also muss man nun nur noch die Zeit, zu der Lisa loslaufen muss, für alle Wege errechnen und den mit der spätesten Losgehzeit auswählen.
Der optimale Weg zu diesen Eckpunkten lässt sich bestimmen mithilfe eines Sichtbarkeitsgraphen und Dijkstra's Algorithmus.

\section{Umsetzung}
Zur Umsetzung habe ich mich für eine Implementation in C\# entschieden, mit einer Visualisierung mithilfe von WPF.



\end{document}